\documentclass{../spanish_summary}

\begin{document}

\chapter*{Futuro Próximo}
Die Zeitform \textbf{Futuro Próximo} drückt Handlungen die in der nahen Zukunft stattfinden werden aus.\\[2ex]
\textbf{Bildung}\\
Die nahe Zukunft wird gebildet durch:\\

\begin{center}
\textit{<Konjugation von ir> + a + <Verb im Infinitiv>}\\[2ex]
\raisebox{.8\baselineskip}{\rotatebox{-90}{$\setlength{\arraycolsep}{.5\arraycolsep}
  \begin{array}{@{}c@{}}
   \aunderbrace[l1D1r]{\begin{array}{rrrrrr}
       \mrot{comer} & \mrot{hablar} & \mrot{apender} & \mrot{ir} &  \mrot{beber} & \mrot{...}  
     \end{array}} \\
   \mrot{~~~a~~~} \hspace{5.3\normalbaselineskip} \\
   \aoverbrace[L1U1R]{\begin{array}{rrrrrr}
       \mrot{voy} & \mrot{vas} & \mrot{vas} & \mrot{vamos} &  \mrot{vais} & \mrot{van}  
     \end{array}}
  \end{array}$}}
\end{center}

Bei reflexiven Verben, werden die Reflexivpronomen (me, te, se, nos, os, se) entweder
\begin{itemize}
  \item vor das Verb ir gestellt oder
  \item an den Infinitiv des Verbes angehangen
\end{itemize}
\bigskip
\textbf{Beispiel}
\begin{itemize}
\item Me voy a duchar = Voy a durcharme
\item Te voy a llamar mana\~{n}a = Voy a llamarte mana\~{n}a
\end{itemize}

\newpage



\end{document}
