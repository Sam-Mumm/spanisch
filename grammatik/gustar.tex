\documentclass[10pt,spanish]{report}
\usepackage[T1]{fontenc}
\usepackage{selinput}
\usepackage{multirow}
\usepackage{tabularx}
\usepackage{titlesec}
\usepackage[none]{hyphenat}
\usepackage{mathtools}
\usepackage{array}
\setlength{\parindent}{0em} 
\newcolumntype{C}[1]{>{\centering\arraybackslash}p{#1}}

\usepackage{abraces,graphicx}

\newcommand{\mrot}[1]{\rotatebox{90}{$#1\mathstrut$}}
\SelectInputMappings{%
  aacute={á},
  ntilde={ñ},
  Euro={€}
}
\usepackage{babel}
\usepackage [left=1.5cm, right=1.5cm, top=1.5cm, bottom=1.5cm]{geometry}
\usepackage{longtable}

\titlespacing*{\section} {0pt}{5.5ex}{0ex}

\begin{document}

\chapter*{Grammatik}
\section*{Präsenz}
\subsection*{reguläre Verben}
\begin{longtable}{p{.15\textwidth} | C{.10\textwidth}  | C{.10\textwidth}  | C{.10\textwidth}} 
\textbf{~}     & hab\textbf{ar} & com\textbf{er} & viv\textbf{ir}                                       \\ \hline
\hline
\endhead % all the lines above this will be repeated on every page
yo & habl\textbf{o} & com\textbf{o} & viv\textbf{o} \\
tú & habl\textbf{as} & com\textbf{es} &  viv\textbf{es} \\
él/ellas/usted &  habl\textbf{a} & com\textbf{e} &  viv\textbf{e} \\
nosotros, -as &  habl\textbf{amos} & com\textbf{emos} &  viv\textbf{imos} \\
vosotros, -as &  habl\textbf{áis} & com\textbf{éis} &  viv\textbf{ís} \\
ellos/ellas/ustedes &  habl\textbf{an} & com\textbf{en} &  viv\textbf{en} \\
\end{longtable}

\subsection*{Vokaländerung (Diphtongierung) }
Für einige \underline{unregelmäßige} Verben gilt:\\[2ex]
$
\left.\parbox{0.45\textwidth}{%
\begin{itemize}
  \item e $\rightarrow$ ie
  \item o $\rightarrow$ ue
  \item e $\rightarrow$ i (\underline{nur} bei Verben der *ir-Konjugation)
\end{itemize}
}\right\} \vbox{\noindent Gilt für alle Formen des Singular und\\für die 3. Person Plural}
$
\subsubsection*{Beispiele}
\begin{longtable}{p{.15\textwidth} | C{.10\textwidth}  | C{.10\textwidth}  | C{.10\textwidth}} 
~ & \textbf{e $\rightarrow$ ie} & \textbf{o $\rightarrow$ ue} & \textbf{e $\rightarrow$ i}\\
~ & querer & poder & pedir                                   \\ \hline
\hline
\endhead % all the lines above this will be repeated on every page
yo & quiero & puedo & pido \\
tú & quieres & puedes &  pides \\
él/ellas/usted &  quiere & puede &  pide \\
nosotros, -as &  queremos & podemos &  pedimos \\
vosotros, -as &  queréis & podéis &  pedís \\
ellos/ellas/ustedes &  quieren & pueden &  piden \\
\end{longtable}

gilt auch für:
\begin{itemize}
  \item dormir
  \item preferir
  \item sentir
  \item so\~{n}ar
  \item decir
\end{itemize}
\newpage

\subsection*{reflexiv Verben}
Drücken Tätigkeiten eines Subjekts aus, die sich auf sich selbst bezieht. Die Bildung erfolgt aus einem Pronom und dem Verb selbst.

\subsubsection*{Beispiel}
\begin{longtable}{p{.05\textwidth} | p{.10\textwidth}} 
\textbf{~}     & llamarse                                      \\ \hline
\hline
\endhead % all the lines above this will be repeated on every page
me & llamo \\
te & llamas \\
se & llama \\
nos & llamamos \\
os & llamais \\
se & llaman \\
\end{longtable}

\subsubsection*{weitere Beispiele}
\begin{itemize}
  \item llamarse \textit{(heißen)}
  \item ducharse \textit{(sich duschen)}
  \item afeitarse \textit{(sich rasieren)}
  \item levantarse \textit{(aufstehen)}
  \item irse \textit{(weggehen)}
  \item quedarse \textit{(bleiben)}
\end{itemize}


\subsection*{Das Verb "gustar"}

$
\left.\parbox{0.1\textwidth}{%
\begin{itemize}
  \item[] me
  \item[] te   
  \item[] le   
  \item[] nos   
  \item[] os  
  \item[] les   
\end{itemize}
}\right\} \vbox{\noindent gusta(n)}
$
\newpage

\section*{Futuro Próximo}
Die Zeitform \textbf{Futuro Próximo} drückt Handlungen die in der nahen Zukunft stattfinden werden aus.\\[2ex]
\textbf{Bildung}\\
Die nahe Zukunft wird gebildet durch:\\

\begin{center}
\textit{<Konjugation von ir> + a + <Verb im Infinitiv>}\\[2ex]
\raisebox{.8\baselineskip}{\rotatebox{-90}{$\setlength{\arraycolsep}{.5\arraycolsep}
  \begin{array}{@{}c@{}}
   \aunderbrace[l1D1r]{\begin{array}{rrrrrr}
       \mrot{comer} & \mrot{hablar} & \mrot{apender} & \mrot{ir} &  \mrot{beber} & \mrot{...}  
     \end{array}} \\
   \mrot{~~~a~~~} \hspace{5.3\normalbaselineskip} \\
   \aoverbrace[L1U1R]{\begin{array}{rrrrrr}
       \mrot{voy} & \mrot{vas} & \mrot{vas} & \mrot{vamos} &  \mrot{vais} & \mrot{van}  
     \end{array}}
  \end{array}$}}
\end{center}

Bei reflexiven Verben, werden die Reflexivpronomen (me, te, se, nos, os, se) entweder
\begin{itemize}
  \item vor das Verb ir gestellt oder
  \item an den Infinitiv des Verbes angehangen
\end{itemize}
\bigskip
\textbf{Beispiel}
\begin{itemize}
\item Me voy a duchar = Voy a durcharme
\item Te voy a llamar mana\~{n}a = Voy a llamarte mana\~{n}a
\end{itemize}

\newpage



\end{document}
