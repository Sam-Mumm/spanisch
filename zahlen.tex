\documentclass{spanish_summary}

\begin{document}

\begin{longtable}{p{.20\textwidth} | p{.50\textwidth}} 
\textbf{Zahlen}     & \textbf{los numeros}                                       \\ \hline
\hline
\endhead % all the lines above this will be repeated on every page
Null (0) & cero \\
Eins (1) & uno \\
Zwei (2) & dos \\
Drei (3) & tres \\
Vier (4) & cuatro \\
Fünf (5) & cinco \\
Sechs (6) & seis \\
Sieben (7) & siete \\
Acht (8) & ocho \\
Neun (9) & nueve \\
Zehn (10) & diez \\
Elf (11) & once \\
Zwölf (12) & doce \\
Dreizehn (13) & trece \\
Vierzehn (14) & catorce \\
Fünfzehn (15) & quince \\
Sechzehn (16) & diez y seis\\
Siebzehn (17) & diez y siete \\
Achzehn (18) & diez y ocho \\
Neunzehn (19) & diez y nueve \\
Zwanzig (20) & veinte \\
Dreißig (30) & treinta \\
Vierzig (40) & cuarenta \\
Fünfzig (50) & cincuenta \\
Sechzig (60) & sesenta \\landinding
Siebzig (70) & setenta \\
Achtzig (80) & ochenta  \\
Neunzig (90) & noventa \\
Hundert (100) & cien \\
Zweihundert (200) & doscientos \\
Dreihundert (300) & trescientos \\
Vierhundert (400) & cuatrocientos \\
Fünfhundert (500) & quinientos \\
Sechshundert (600) & seiscientos \\
Siebenhundert (700) & setecientos \\
Achthundert (800) & ochocientos \\
Neunhundert (900) & novecientos \\
Tausend (1000) & mil \\

\end{longtable}

\newpage

\begin{longtable}{p{.20\textwidth} | p{.50\textwidth}} 
\textbf{Ordnungszahlen}     & \textbf{los números ordinales}                                       \\ \hline
\hline
\endhead % all the lines above this will be repeated on every page
erste, -r & primero, -a\\
zweite, -r & segundo, -a\\
dritte, -r & tercero, -a\\
vierte, -r & cuarto, -a\\
fünfte, -r & quinto, -a\\
sechste, -r & sexto, -a\\
siebte, -r & séptimo, -a\\
achte, -r & octavo, -a\\
neunte, -r & noveno, -a\\
zehnte, -r & décimo, -a\\
elfte, -r & undécimo, -a\\
zwölfte, -r & duodécimo, -a\\
dreizehnte, -r & decimotercero, -a\\
vierzehnte, -r & decimocuarto, -a\\
fünfzehnte, -r & decimoquinto, -a\\
sechzehnte, -r & decimosexto, -a\\
siebzehnte, -r & decimoséptimo, -a\\
achtzehnte, -r & decimooctavo, -a\\
neunzehnte, -r & decimonoveno, -a\\
zwanzigste, -r & vigésimo, -a



\end{longtable}


\end{document}
